\section{Estado da Arte}

No ramo de estudo de lesões na pele, as soluções podem ser classificadas de acordo com o método empregado para encontrá-los. Alguns métodos são por inteligência artificial, outros por limiarização, ou clusterização ou modelos deformávieis. Em todos eles cabe uso automático ou não dos mesmos, ou seja, pode haver um especialista escolhendo amostras ou ser integralmente independente da ação humana.

Com auxílio do \cite{arxiv}, foi pesquisdao modelos posteriores à 2019 pelas seguintes pesquisas: \textit{melanoma segmentation}, \textit{skin lesions segmentation}, \textit{melanoma auxiliary image segmentation}, todos no ramo de ciência da computação. Dessa forma, o objetivo com a pesquisa dos métodos já existentes é entender as barreiras existentes e quais resultados são possíveis, uma vez que este artigo visa implementar um método que também será analisado na sequência.

\begin{table}[]
  \begin{tabular}{llll}
    Palavra chave           & Frequência & Semi-automático & Orientado à Performance \\
    Inteligência Artificial & 13         & 0               & 4                       \\
    Limiarização            & 2          & 1               & 1                       \\
    Contornos Deformáveis   & 2          & 0               & 1                       \\
    Clusterização           & 4          & 2               & 2                       \\
    TOTAL                   & 21         & 3               & 8
  \end{tabular}
  \caption{Frequência e atributos de artigos com as palavras chaves comuns}
  \label{table:state-of-art}
\end{table}

A tabela \ref{table:state-of-art} expressa os métodos comuns no ramo de segmentação de lesões e, principalmente, do melanoma. Embora o método deste artigo não seja o mais frequente pelas pesquisas realizadas. É importante verificar que o método de inteligência artificial apresentou o menor número de orientação à performance, bem como é importante levar em consideração o viés de maturação, isto é, o fato de métodos semi-automáticos ou não baseados em inteligência artificial terem menor quantidade de artigos escritos recentemente pode implicar no fato de que estes métodos já são mais maduros, independentemente da sua performance, ou seja, são métodos que alcançaram bons resultados e não se tem uma visão clara de pontos possíveis para melhoras, enquanto métodos de inteligência artificial ou automáticos, por mais que tenham ótimos desempenhos, precisam de ainda mais pesquisa e desenvolvimento para avançar a fronteira do conhecimento.

O artigo \cite{santos2020skin} que será usado como base para implementação de um método semi-automático de segmentação de lesões de pele utilizou métodos de pré-processamento baseado em superpixel, clusterização em \emph{Fuzzy C-means} e extração de descritores de textura, era baseado na escolha semi-automática de superpixels e utilizava apenas datasets públicos. Foi utilizado o \emph{Métrica do índice de Jaccard} (\emph{TJI}), e teve acurácias (mínimas) que variavam de 65\% à 94.28\%.

Observado os métodos recentes e do artigo que oferece as bases para implementação neste artigo, verifica-se, portanto, que o modelo é baseado em métodos mais sólidos que os mais frequentes atualmente, entretanto, o principal objetivo é atestar se é possível com simplificações do método proposto \cite{santos2020skin} manter o \emph{TJI} nos valores previstos ou superiores.

