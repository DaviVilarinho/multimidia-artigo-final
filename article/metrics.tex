\subsection{Datasets}

Um fator muito importante na avaliação de imagens é a fonte de dados envolvida para testes e desenvolvimento. Como não é necessário treinar e testar em fases distintas, este método exige apenas imagens de lesões e a respectiva \textit{``ground truth''} para verificação.

Serão utilizadas as imagens do dataset público ISIC 2016 \cite{isic2016} para testar a eficiência de cada particionamento, os resultados serão comparados à \textit{``ground truth''} também disponibilizada.

TODO aspectos de cor e tamanho do dataset

\subsection{Métricas de Avaliação}

As métricas de avaliação, para fins de compatibilidade de resultados, serão todos iguais ao \cite{santos2020skin}, ou seja: 'positivo' será usado para designar áreas que são lesão e 'negativo' as que não são, serão consideradas métricas como acurácia, especificidade, sensitividade e Métrica do Índice de Jaccard.

Os resultados não tem como objetivo superar as métricas já existentes no artigo base, mas sim indicar quais apresentaram melhor pontuação inicial, uma vez que o objetivo é melhorar a imagem disponibilizada ao especialista, e não necessariamente a segmentação final.

\subsection{Resultados}

O método requer passar o parâmetro de superpixel que são o intervalo de 100 a 1600 acrescidos de 50. Em seguida serão realizados os passos descritos na seção de metodologia e a imagem obtida da redução ao Otsu será avaliada com as métricas descritas.

TODO resultados
